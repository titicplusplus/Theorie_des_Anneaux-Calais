\documentclass[a4paper,12pt]{report}
\usepackage[french]{babel}
\usepackage[utf8]{inputenc}
\usepackage[T1]{fontenc}
\usepackage{tikz}
\usepackage{titlesec}
\usepackage{hyperref}
\usepackage{pgfmath}
\usepackage{amsmath} % Pour des outils mathématiques avancés
\usepackage{amsfonts}

\hypersetup{
    colorlinks=true,
    linkcolor=purple,
    urlcolor=cyan,
    pdftitle={Corrigés Mathématiques},
}

% --- Configuration des titres ---
\titleformat{\section}{\bfseries\Large}{\thesection}{1em}{}
\titleformat{\subsection}{\bfseries\normalsize}{\thesubsection}{1em}{}

% --- Macros pour inclure les exercices ---
\newcommand{\includesection}[1]{
    \section*{#1}
    \addcontentsline{toc}{section}{#1}
}

\newcommand{\includeexercise}[1]{
    \subsection*{Exercice #1}
    \addcontentsline{toc}{subsection}{Exercice #1}
    \input{#1}
}

\begin{document}

% --- Page de garde ---
\begin{titlepage}
    \begin{center}
        \vspace*{2cm}
        {\Huge\bfseries Anneaux - Corps : Eléments de théorie des anneaux, correction \\[1cm]}
        {\Large Année : \the\year \\[1cm]}
        \textbf{Nom des Correcteurs :} \\[0.5cm]
        Thibaud VINCENT \\[4cm]

        \textit{L3 Mathématiques}
    \end{center}
\end{titlepage}
\newpage

% --- Page des progrès des exercices ---

% Définir le pourcentage d'exercices réalisés
\def\totalexercises{112}
\def\completedexercises{3}

\pgfmathsetmacro{\percentage}{\completedexercises/\totalexercises*100}
\pgfmathsetmacro{\remainingpercentage}{100-\percentage}

\section*{Progrès des exercices}


% Affichage du cercle avec le pourcentage et deux couleurs (noir et gris)
\begin{center}
    \begin{tikzpicture}
        % Portion de 60% en noir pour l'épaisseur du cercle
        \draw[line width=20pt, black] (0:5cm) arc[start angle=0, end angle=\percentage*3.6, radius=5cm];
        
        % Portion de 40% en gris pour l'épaisseur du cercle
        \draw[line width=20pt, gray] (\percentage*3.6:5cm) arc[start angle=\percentage*3.6, end angle=360, radius=5cm];

        % Affichage du pourcentage au centre du cercle
        \node at (0,0) {
            \textbf{\huge \percentage\% - \completedexercises / \totalexercises }
       };
    \end{tikzpicture}
\end{center}


% Affichage du nombre d'exercices réalisés par chapitre

\textbf{Progrès par chapitre}
\begin{enumerate}
    \item Structure d'anneau  \dotfill 3 / 16
    \item Idéaux d'un anneaux \dotfill 0 / 18
    \item Modules et Algèbres \dotfill 0 / 10
    \item Algèbres de polynômes \dotfill 0 / 9    
    \item Factorisation dans les domaines d'intégrité  \dotfill 0 / 29
    \item Localisation   \dotfill 0 / 10
    \item Séries formelles  \dotfill 0 / 7
    \item Polynômes symétriques  \dotfill 0 / 13
\end{enumerate}    


\newpage

% --- Sommaire ---
\tableofcontents
\newpage

% --- Inclusion des exercices ---
\foreach \chapter in {1} { % Parcours des chapitres
    \section*{Chapitre \chapter}
    \addcontentsline{toc}{section}{Chapitre \chapter}
    \foreach \exercise in {1, 2, 3} { % Parcours des exercices
        \IfFileExists{Chapitre\chapter/Exercice-\chapter-\chapter.\exercise.tex}{%
            \subsection*{Exercice \exercise}
            \addcontentsline{toc}{subsection}{Exercice \exercise}
            \input{Chapitre\chapter/Exercice-\chapter-\chapter.\exercise.tex}
        }{}
    }
    
    \newpage
}

\end{document}
