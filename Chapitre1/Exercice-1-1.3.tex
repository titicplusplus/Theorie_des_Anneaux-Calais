% Exercice 1.1

Vérifions que $\mathbb{Z}[i\sqrt{5}]$ est un sous anneaux de $\mathbb{C}$

Déjà on a $\mathbb{Z}[i\sqrt{5}] \subset \mathbb{C}$
\newline
$\forall (x, y) \in \mathbb{Z}[i\sqrt{5}], \exists (a, b, c, d) \in \mathbb{Z}^4\ |\ x = a + i\sqrt{5}b, y = c + i\sqrt{5}d$

$x - y = a + ib - c - id = (a - c) + i\sqrt{5}(b - d), a - c \in \mathbb{Z}, b - d \in \mathbb{Z}$

Donc $x - y \in \mathbb{Z}[i\sqrt{5}] \Longrightarrow (\mathbb{Z}[i\sqrt{5}], +)$ est un groupe additif (et abélien car $\mathbb{Z}[i\sqrt{5}] \subset \mathbb{C}$)

$xy = (a + ib)(c + id) = ac - bd + i\sqrt{5}(bc + ad) \in \mathbb{Z}[i\sqrt{5}]$, la multiplication est donc bien interne.

$1 \in \mathbb{Z}[i\sqrt{5}]$ et 1 est un élement d'unité de $\mathbb{C}$

$ \Longrightarrow  \mathbb{Z}[i\sqrt{5}]$ est un sous-anneaux unitaire.

$$ $$
Même démonstration pour montrer que  $\mathbb{Q}[i\sqrt{5}]$ est un sous-anneaux unitaire de $\mathbb{C}$.

$\mathbb{Q}[i\sqrt{5}]$ est commutatif car $\mathbb{Q}[i\sqrt{5}] \subset \mathbb{C}$ et $\mathbb{C}$ commutatif.


Un anneaux commutatif unitaire est un corps si $\mathbb{Q}[i\sqrt{5}]^{\times} = \mathbb{Q}[i\sqrt{5}] \setminus \{0\}$

$\forall a + i\sqrt{5}b \in \mathbb{Q}[i\sqrt{5}] \setminus \{0\}$

\[
  \frac{1}{a + i\sqrt{5}b} = \frac{a - i\sqrt{5}}{(a + i\sqrt{5}b)(a - i\sqrt{5}b)} = \frac{a - i\sqrt{5}}{a^2 + 5b^2} = \frac{a}{a^2 + 5b^2} - \frac{i\sqrt{5}}{a^2 + 5b^2} \in \mathbb{Q}[i\sqrt{5}]
\]

Et 

\[
  (a + i\sqrt{5}b)*\frac{a - i\sqrt{5}}{a^2 + 5b^2} = \frac{a^2 + 5b^2}{a^2 + 5b^2} = 1 \Longrightarrow (a + i\sqrt{5}b) \in \mathbb{Q}[i\sqrt{5}]^{\times}
\]

Donc $\mathbb{Q}[i\sqrt{5}]^{\times} = \mathbb{Q}[i\sqrt{5}] \setminus \{0\} \Longrightarrow \mathbb{Q}[i\sqrt{5}]$ est un corps
